\RequirePackage{plautopatch}
\RequirePackage[l2tabu, orthodox]{nag}

\documentclass[platex,dvipdfmx]{jlreq}			% for platex
% \documentclass[uplatex,dvipdfmx]{jlreq}		% for uplatex
\usepackage{graphicx}
\usepackage{bxtexlogo}
\usepackage{ascmac}
\usepackage{fancybox}
\usepackage{url}%%% パッケージ url を読み込む


\title{弱双対定理・双対定理}

\author{T-N}
\date{\today}
\begin{document}
\maketitle
\section{2段階法}

前のセクションで述べた単体法は、実行可能基底解が存在するという仮定の下、最適化の作業を進めていったが、実際の最適化問題では「そもそも制約条件が矛盾していて、実行可能解が存在しない」「実行可能基底解を標準形から容易に見つけることができない」ような事例に遭遇することがある。
\\
そのような場合には、\textbf{2段階法}を用いる。\\
2段階法の概要は以下のとおり。

\begin{itembox}[l]{2段階法}
1. 人為変数$v_1, v_2, \cdots  v_n$を導入する。
(この時、$v_1, v_2, \cdots  v_n$は実行可能基底解になっている)\\
2. これらの人為変数の合計が0にできるような$x_1, x_2, \cdots x_n$が存在すれば、その$x$の組み合わせが実行可能基底解になる。\\
(人為変数の合計が0になるような組み合わせが存在しない場合は、実行可能基底解が存在しないことになるので、その旨を出力しておしまい)\\
3. 上記で求めた$x$をもとに単体法を行い、最適解を求める。
(ここからは普通の単体法と同じ)
\end{itembox}



\section{巡回}
実は、退化した制約条件においては有限個(${}_n C_r$個)の実行可能基底解を調べるだけで最適解を求めることができる。
\\
\begin{itembox}[l]{退化していない単体法の収束性}

非退化の仮定の下では単体法は有限回の反復で終了する。\\
(最悪でも${}_n C_r$回の操作をすれば最適解を求めることができる)
\end{itembox}



上記のような$\textbf{巡回}$を防ぐためには$\textbf{Blandの巡回対策法}$を用いる。
この方法では、掃き出しのピボットを選択する際に最小の添字を選ぶことで添字の選び方を一意にしている。


以下、参考文献(3)より引用\\
目的行において正係数を持つ添字番号最小の変数を取入れ,ratio-test のとき最小と
なる基底変数が複数あるときは,添字番号が最小のものを追い出す.

\section{双対性}

線型計画問題の標準形

最小化 $\cdots$  $ f_p=c_1x_1+c_2x_2+c_3x_3+\cdots+c_nx_n $ \\
制約条件 $\cdots$ $ Ax=b $  \quad  $(x \geq0)$

に対して、下記の問題を$\textbf{双対問題}$と呼ぶ。\\

最大化 $\cdots$ $ f_d=c_1x_1+c_2x_2+c_3x_3+\cdots+c_nx_n $ \\
制約条件 $\cdots$  $A^T y \leq  c$
\\
$ f_p=c_1x_1+c_2x_2+c_3x_3+\cdots+c_nx_n $の最小化を \textbf{主問題}、
$ f_d=c_1x_1+c_2x_2+c_3x_3+\cdots+c_nx_n $の最大化を  \textbf{双対問題}と呼びます。


\subsection*{弱双対定理}
\begin{itembox}[l]{弱双対定理}
$\bar{x}, \bar{y}$がそれぞれ、主問題(P),双対問題(D)の実行可能解であれば以下が成立する。

$
\bar{f_p}  =  c^T \bar{x}  \geq  b^T \bar{y} =  \bar{f_d}
$

\end{itembox}

[証明]\\
$\bar{x}と\bar{y}$は実行可能解なので、\\
(1)$\quad A \bar{x} = b \quad$  $(x \geq 0)$ \\
(2)$\quad A^T \bar{y}  \leq  c $ \\
を満たす。
\\
$
\bar{f_p}  =  c^T \bar{x}  \geq  (A^T \bar{y})^T \bar{x}  \quad ←(2)より \\
 (A^T \bar{y})^T \bar{x} =  (\bar{y})^T (A \bar{x})  =  b^T \bar{y} =  \bar{f_d}
$
\\
よって
$
\bar{f_p}  \geq   \bar{f_d}
$
が成立する。
\\
また、弱双対定理から下記の事実が得られる。

\begin{itembox}[l]{系1.1}

(1)主問題の実行可能解$\bar{x}$と双対問題の実行可能解$\bar{y}$に対して
$c^T \bar{x} = b^T \bar{y}$が成り立つならば
$\bar{x}$と$\bar{y}$は、それぞれ主問題・双対問題の最適解になる。

(2)主問題の目的関数が下に有界ではない、つまり最小値が$- \infty $になる場合、もしくは双対問題の目的関数が上に有界ではない、最大値が$\infty$になるならば、他方の問題は実行可能ではない。
\end{itembox}
[証明]\\
(1)弱双対定理より、
$
\bar{f_p}  \geq   \bar{f_d}
$
が成立する。\\
よって、 $\bar{f_p}$には最小値が、$\bar{f_d}$には最大値が存在することがわかる。\\
($\bar{f_p}は \bar{f_d}よりも大きいため、最小値が存在。 また、\bar{y}の最大の値は\bar{x}で抑えることができている$)\\
$c^T \bar{x} = b^T \bar{y}$が成立しているとき、最小値=最大値をみたしているため、これを満たす$\bar{x} , \bar{y}$は最適解になるとわかる。
\\
(2)主問題の目的関数が下に有界ではないとする。\\
双対問題に実行可能解$\bar{y}$が存在すると仮定すると、弱双対定理より
$c^T x  \leq b^T \bar{y}$ が成立する。
しかし、仮定より$c^T x → -\infty$となるため、不等号が成立しなくなってしまい不適。\\
よって双対問題に実行可能解は存在しない。逆の場合も同様に示すことができる。


\subsection*{双対定理}
\begin{itembox}[l]{双対定理}
(1)双対問題が最適解を持つならば、それに対応する単体乗数が双対問題の最適解になる。
さらに、双対問題・主問題の最適値は等しくなる。\\
(2)主問題、双対問題が両方とも実行可能解を持つならば、どちらも最適解を持ち、それぞれの最適値が一致する。
\end{itembox}
[証明]\\
主問題の最適基底解を$x^*$として、その時の基底行列を$B$, 基底変数ベクトルを$x^*_B$とすれば、最適性基準より(→P56 $\bar{c}_N \leq 0$)下記が成り立つ。\\
$
B_{*} x^*_B = b \\
\bar{c}_N ^*  =  c_N  -  N^T_{*}  y^*  \leq  0
(yは最適解に対応する単体乗数であり、B_{*} ^T y^*  =  c_Bを満たす)
\\
A^T y^* \geq c を満たすのでy^*は双対問題の実行可能解になる。(双対問題の制約条件を満たしている)\\
この時、目的関数値は\\
f_{d} ^* = b^T y^* = 
$
よって系1.1(1)より$y^*$は双対問題の最適解になる。\\
($f_{d}$と$f_{p}$が等しいため)\\
(2)主問題・双対問題の両方が実行可能解を持つので、弱双対定理が成立する。
系1.1(2)より、主問題・双対問題が実行可能であるときには、下記の2つが同時に成り立つ。\\
(1)主問題の目的関数が下に有界\\
(2)双対問題の目的関数が上に有界\\
よって、両方の問題は最適解を持つため、(1)と同様に示すことができる。

\section*{次回予告}
次回は、双対定理の対称性・相補性定理の説明をしたのち、双対単体法について解説していく予定です。

\section{参考文献}

(1).2段階シンプレックス法.\\
\url{https://www2.kaiyodai.ac.jp/~yoshi-s/Lectures/Optimization/2009/lecture_7.pdf}\\
 \quad (2)矢部博. (2006). 工学基礎 最適化とその応用 (8th ed.). 数理工学社.\\
 \quad(3)Biglobe.巡回について.
\\
\\
\section*{さいごに}
*万が一、誤植や誤りなどがあった際には教えていただけると幸いです。

\end{document}
